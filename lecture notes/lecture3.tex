\hrule
\vspace{0.1in}
\noindent
\textbf{Lecture Topic: } Verification of Matrix Multiplication and Bayes Theorem\\
\textbf{Lecture Date: }08/27/2024 \\
\textbf{Scribe Authors: }Dhruv Trivedi, Jeremy Easton-Marks
\vspace{0.1in}
\hrule
\vspace{0.1in}

This lecture continued our work on the verification of matrix multiplication and discussed Miller Rabin. We also discussed Bayes Theorem

\section*{Prime Number Testing}
How do we efficiently test if $n$ is a prime number?

We can calculate this naively by checking if it is divisible from range $2$ to $\sqrt{n}$. This is not a polynomial. We can use Miller Rabin's test to reduce the cost to $\bigO(\log^3n)$. This algorithm is still in use and relies upon randomization. \\

Another algorithm that we can us is the AKS algorithm was created in 2021 which has an $\bigO(\log^{12}n)$. This algorithm is deterministic but is slightly slower.

\begin{theorem}[Fermat's Little Theorem]
A way of determining if a number $p$ is prime or not

\begin{enumerate}
\item If p is prime then it is correct. (mod p)
\item If p is complement is $2^{\frac{1}{100}}$ (mod p) then p is prime.
\end{enumerate}

\end{theorem}

\begin{theorem}[Fermat's Factorization Method]
If $n=x^2 - y^2$ then $n=(x-y)(x+y)$

\end{theorem}

\section*{Miller Rabin's Test}
Miller Rabin's test is based on Fermat's Little Theorem


\[
\forall 1 \geq a \geq n a^{n - 1} \equiv 1 (\mod n)
\]

The basic algorithm is to take a random number from the sample space $(1...n - 1)$. Mod the number and then square the number. Repeat this until the square of the mode is equal to mod then square.


$\forall prime a \equiv 1 \mod n$ than
$\sqrt{a} \equiv \pm 1 \mod n$


\section*{Bayes Theorem}
Given $a$ -> event is a is prime and $b$ -> event $N_R$ returns prime. We want to calculate probability of $A$ given $B$ or $Pr(A|B)$.

\[
\begin{aligned}
Pr(A|B) & = \frac{Pr(A \cap B }{Pr(B)} \\
Pr(A|B) & = \frac{Pr(B|A)Pr(A)}{Pr(B)} \\
A(B) & = Pr(B|A)Pr(A) + Pr(B|A)Pr(B)
\end{aligned}
\]

Prime Number Theorem -> For n apex have more are prime. $\frac{1}{2n(n)}$ primes

\[
\begin{aligned}
Pr(A) & =\frac{1}{\ln} \\
Pr(B) & =\frac{1}{2^1000} \\
Pr(\bar{A}) & = 1 - \frac{1}{\ln}
\end{aligned}
\]

\[
\begin{aligned}
& \frac{\frac{1}{\ln}}{ \frac{1}{\ln} + 2 ^{-100} ( 1 - \frac{1}{\ln} )} \\
& \frac{1}{1+ 2^{-100}(\ln-1)} \\
& \frac{1}{1+ 2^{-\log n}(\ln-1)} \\
& \frac{1}{1+ \frac{1}{n}(\ln-1)} \\
\end{aligned}
\]

\section*{Min Cut Problem -> Karger's Algorithm}
Given a graph $G$ find the least number of cuts to break a graph in $\frac{1}{2}$.

$m \in M$ where $M$ is Edges \\
$n \in N$ where $N$ is Nodes

A deterministic algorithm takes $O(mn)$ while a random sampling can reach $0(n^2\log^3n\lim(n))$.

Karger's Algorithm allows multi-edges and random sampling.

$Pr($min cut is output$)$ what is the success probability
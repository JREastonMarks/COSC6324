% Everything before `\begin{document}` is called the preamble. The preamble for this document is
% included in the `main.sty` file and is imported into the main document. There is usually no need to
% change anything before the \begin{document} command. However, you can add things as needed, 
% such as additional commands and new environments.

\documentclass[a4paper,11pt,final]{article}
\usepackage{scribe}

\begin{document}

%%%%%%%%%%%%%%%%%%%%%%%%%%%%%%%%%%%%%%%%%%%%%%%%%%%%%%%%%%%%%%%%%%%%%%%%%%%
% Document begins here %%%%%%%%%%%%%%%%%%%%%%%%%%%%%%%%%%%%%%%%%%%%%%%%%%%%
%%%%%%%%%%%%%%%%%%%%%%%%%%%%%%%%%%%%%%%%%%%%%%%%%%%%%%%%%%%%%%%%%%%%%%%%%%%

% Keep everything up to here. Start writing your notes from here.

\begin{center}
\textbf{\large COSC 6324 \\ Randomized Algorithms and Probabilistic Techniques in Computing} \\

Professor Gopal Pandurangan\\
\vspace{0.1in}
\end{center}

\hrule

\vspace{0.1in}

\textbf{Lecture Topic:} TOPIC \\

\textbf{Lecture Date:} DATE \\



\textbf{Scribe Authors:} All names in your team

\vspace{0.1in}

\hrule

\vspace{0.1in}

In this lecture, we recall basic definitions and concepts of probability theory. 

\section{Events, Probability, Probability Space}
Consider an experiment with a finite (or countably infinite)
number of outcomes. Each outcome is a simple event (or a sample point).
The \emph{sample space} is the set of all possible simple events.
An \emph{event} $\mathcal{E}$ is a union of simple events --- a subset of the sample space.

Two events are said to be \textbf{mutually exclusive} if $A \cap B = 0$.

With each simple event $s$ we associate a number $\prob{s}$ which is
called as the \emph{probability} of $s$.

\begin{definition}[Probability Space]
    A probability distribution $\Pr$ on a
    discrete sample space $S$ is a function from events of $S$ to $\mathbb{R}$
    such that it satisfies the following three \emph{axioms}:
    \begin{enumerate}
        \item $\prob{A} \geq 0$ for any event $A$.
        \item $\prob{S} = 1$.
        \item For any (finite or countably infinite)  sequence of pairwise mutually exclusive events
              $A_1, A_2, \dots$:
              \[\prob{\cup_i A_i} = \sum_i\prob{A_i}\]
    \end{enumerate}
    The pair $(S, \Pr)$ is called a discrete \textbf{probability space}.
\end{definition}

The probability of any event $\mathcal{E}$  can be computed as the sum of the probabilities of the simple events
that it is composed of; this follows from the third axiom:
\[\prob{\mathcal{E}} = \sum_{s\in\mathcal{E}} \prob{s}\]


\section{Principle of Inclusion-Exclusion}

The \emph{inclusion-exclusion} principle gives a formula for computing the probability
of the union of a set of events in terms of the probabilities of the individual events and their intersections.

\begin{theorem}[Inclusion-Exclusion Principle]
    Let $E_1,E_2, \dots, E_n$ be arbitrary events. Then
    \begin{align*}\prob{\cup_{i=1}^nE_i}
         & = \sum_{i=1}^n\prob{E_i} - \sum_{1 \leq i < j\leq n}\prob{E_i \cap E_j} + \sum_{1 \leq i < j < k \leq n}\prob{E_i \cap E_j \cap E_k} \\
         & \phantom{=} - \dots + (-1)^n\prob{E_1 \cap E_2 \dots \cap E_b}                                                                       \\
         & = \sum_{k=1}^n(-1)^{k+1}\left(\sum_{1\leq i_1<\dots<i_k\leq n}\prob{E_{i_1}\cap E_{i_2}\cap\dots\cap E_{i_k}}\right)
    \end{align*}
\end{theorem}

An important consequence of the inclusion-exclusion principle is a simple upper bound on the probability
of the union of events known as \emph{Boole's inequality or union bound}.

\begin{theorem}[Boole's inequality (union bound)]\label{thm:boole}
    For any
    arbitrary sequence of events $E_1, E_2, \dots, E_n$:
    \[\prob{\cup_{i=1}^nE_i} \leq \sum_i \prob{E_i}\]
\end{theorem}

The union bound is used often in upper bounding the occurrence of the union of a set of ``bad'' events. If this upper bound is small,
then it implies that none of the bad events occur with high probability.


\section{Conditional Probability}

\begin{definition}[Conditional Probability]
    Given two events $A$ and $B$, the conditional probability of $A$ given $B$ is defined as follows:
    \[\prob{A \given B} = \frac{\prob{A\cap B}}{\prob{B}}\]
    assuming $\prob{B} \neq 0$.
\end{definition}

In the above definition, by conditioning on $B$ we restrict the sample space to the set $B$.
Thus the conditional probability can be considered as  $\prob{A \cap B}$ ``normalized'' by $\prob{B}$.

Given two events $E_1$ and $E_2$, Bayes' rule relates the conditional probability of the first given the second,
to the conditional probability of the second given the first. This is useful to infer one conditional probability from the other.
\begin{theorem}[Bayes' rule]
    \[\prob{E_1 \given E_2}=\frac{\prob{E_1 \cap E_2}}{\prob{E_2}} = \frac{\prob{E_2 \given E_1}\prob{E_1}}{\prob{E_2}}\]
\end{theorem}

The following are some useful identities that involve computing the probability of the intersection of many events.

\khalid{When you compile this theorem looks ugly. Not clear why?}
\begin{theorem}[Chain rule of Conditional Probabilities]\label{thm:chainrule}
    \begin{itemize}
        \item $\prob{A\cap B}=\prob{A \given B} \prob{B}$.
        \item $\prob{A\cap B\cap C}=\prob{A \given B\cap C}\prob{B\cap C} = \prob{A \given B\cap C} \prob{B \given C}\prob{C}$.
        \item The following is a generalization of the above identity.
              Let $A_1$, $\dots$, $A_n$ be a sequence of events.
              Let $E_i = \bigcap_{j=1}^i A_i$. Then
              \begin{align*}\prob{E_n}
                   & =\prob{A_n \given E_{n-1}}\prob{E_{n-1}}                                                      \\
                   & = \prob{A_n \given E_{n-1}}\prob{A_{n-1} \given E_{n-2}}\dots \prob{A_2 \given E_1}\prob{A_1}
              \end{align*}
    \end{itemize}
\end{theorem}

A fundamental concept that follows from conditional probability is that of \emph{independence}.

\begin{definition}[Independence of events]
    Two events $A$ and $B$ are  said to be \emph{independent} if
    \[\prob{A\cap B} = \prob{A} \times \prob{B}\]
    or (when $\prob{B} > 0$)
    \[\prob{A \given B} = \frac{\prob{A\cap B}}{\prob{B}} = \prob{A}\]
\end{definition}


\section{The Birthday Paradox}

We study a problem called the ``Birthday Paradox'', that arises often in algorithm design and analysis.
The problem also serves to illustrate basic concepts in probability theory.

Question: What is the probability that among $m$ people no two have the same birthday?
We make the  following two assumptions: (1) All birthdays are equally likely and (2) Birthdays are independent events.

To compute the above probability, we first define the sample space of the experiment.
The sample space is the set of all vectors $S = \{(b_1, \dots, b_m) | b_i \in [1,
        \dots, N]\}$, where $b_i$ denotes the birthday of the $\nth{i}$ person, and $N$ is the total number of different birthdays ($N= 365$ in Earth).
We need to compute $\prob{E}$ where the event $E = \{(b_1,\dots, b_m | b_i \neq b_j \mbox{ for all }
    i \neq j\}$, i.e., the event $E$ is the set of all events where no two birthdays are the same.

How many different atomic events are counted in $E$?
The number of possible $m$ different birthdays is $N.(N-1).(N-2) \dots (N-m+1)$.
Hence,
\begin{align*}\prob{E}
     & = \frac{N\cdot(N-1)\cdot(N-2)\dots(N-m+1)}{N^m} \\
     & = \prod_{i=0}^{m-1}(1 - i/N)                    \\
     & \leq \prod_{i=0}^{m-1} e^{-i/N}                 \\
     & = e^{-\sum_{i=0}^{m-1} i/N}                     \\
     & = e^{-m(m-1)/2N}
\end{align*}
For $m = \sqrt{2N} + 1 \leq 28$, $\prob{E} < 1/e < 1/2$.

The apparent ``paradox'' in this problem is that significantly less number of people, i.e.,  only about $\sqrt{N}$ people (which is much smaller than $N$), are needed to have a good chance (about 50\%) to have two people to have a common birthday.

\subsubsection*{Alternate Analysis}
Assume that we choose one birthday after the other independently and uniformly at
random from $[1, 2, \dots, N]$.
Let the event $E_i$ denote: ''the $\nth{i}$ choice is different from the first $i-1$ choices''.
Then, we compute:
\begin{align*}\prob{E}
     & = \prob{\cap_{i=1}^mE_i}                                                                                     \\
     & = \prob{E_1} \prob{E_2 \given E_1}\prob{E_3 \given E_2 \cap E_1} \dots \prob{E_m \given \cap_{i=1}^{m-1}E_i} \\
     & = 1 (1 - 1/N) (1 - 2/N) \dots (1 - (m-1)/N)                                                                  \\
     & = \prod_{i=1}^{m} (1 -\frac{i-1}{N})
\end{align*}
This gives the same result as before.

This analysis uses a very useful principle called the  \textbf{Principle of Deferred Decisions}. In this principle,
the idea is to defer the fixing of choices of events to ``when they are needed'' and not a priori (i.e., all at once).
Typically, this type of analysis will use conditional probabilities. In the above analysis, to compute $\prob{E}$,
we first computed $\prob{E_1}$, then $\prob{E_2 \given E_1}$ and so on. At each stage, we only fixed the  event that
determined the respective conditional probabilities, i.e., first $E_1$ which fixes $b_1$, and then the event
$E_2$ conditioned on $E_1$, which fixes $b_2 \neq b_1$, and so on.



% For readability, it is good to write each chapter/section in its own .tex file
% and input them into the main document. See below.


\section{Latex resources}


An easy way to write latex is using overleaf:  \url{http://overleaf.com}.

See the latex tutorials at \url{https://www.overleaf.com/learn/latex/Tutorials}

You can also find various latex manuals and documentation at
\url{https://latex-project.org/guides/}.
\section{Theorems}



Theorems that we discuss in class should be stated correctly and proved. Here is an example.

\bigskip

Markov's inequality uses only the value of the expectation (also called ``first moment'').
Hence it can be quite weak.

\begin{theorem}[Markov Inequality]
    For any non-negative random variable and for any $a > 0$
    \[\prob{X\geq a}\leq \frac{\expectation{X}}{a}\]
\end{theorem}


Except for the easiest theorems and proofs, it is important to give an intuition for them.
Here is an example.

Intuitively, Markov's inequality says that for any (positive) random variable $X$, it is less likely  that
$X$ takes a value  much larger than its expectation. 



\begin{proof}
    For any $a > 0$, let
    $I$ be an indicator r.v. for the event  $X \geq a$.
    Then $I \leq X/a$. Taking expectations on both sides, we get   $\expectation{I} = \prob{X \geq a} \leq \frac{\expectation{X}}{a}$.
\end{proof}



Chebyshev's inequality gives a stronger bound, which  assumes that the \emph{variance} (also called the ``second moment'') or \emph{standard deviation} of the random variable is known.
These are defined below.


It is good to define essential concepts formally by defining them. Here is an example.

\begin{definition}
    The \textbf{variance} of a random variable $X$ is
    \[\variance{X} = \expectation{\left(X-\expectation{X}\right)^2}\]
    The \textbf{standard deviation} of a random variable $X$ is
    \[\sigma(X)=\sqrt{\variance{X}}\]
\end{definition}

Now we state and prove Chebyshev's inequality.

\begin{theorem}{Chebyshev's Inequality}
    For \textbf{any} random variable
    \[\prob{\abs{X-\expectation{X}} \geq a} \leq \frac{\variance{X}}{a^2}\]
\end{theorem}

The proof uses Markov's inequality. Note that Chebyshev's inequality applies to any random variable,
not necessarily positive. Thus to apply Markov (which applies only to positive r.v.) we 
apply it to $\abs{X-\expectation{X}}$. The usual trick to get rid of absolute value, which is hard to deal with, is by squaring.


\begin{proof}
    \[\prob{\abs{X-\expectation{X}} \geq a} = \prob{(X-\expectation{X})^2 \geq a^2}\]
    By Markov inequality,
    \[ \prob{(X-\expectation{X})^2 \geq a^2} \leq \frac{\expectation{\left(X-\expectation{X}\right)^2}}{a^2}=\frac{\variance{X}}{a^2}\]
\end{proof}


\section{A Lemma in Latex}

Lemmas are statements that help in proving a theorem (usually the main result).
Lemmas also should be stated correctly and proved. This Lemma is from one of my papers.

\begin{lemma}
\label{lem:capacity} Let $C> 4D$, then at any given step $t \geq c\log N$, for a fixed
constant $c >0$, w.h.p. there is at least $\frac{1}{4}c\log N$ d-nodes in the network.
Further, the above set of d-nodes arrived during the interval $[t-c\log N, t]$.
\end{lemma}

\begin{proof}
Let $v$ be a node of degree $d(v)$ at time $\tau$, then the expected number of
neighbors of $v$ that leave the network at time $\tau$ is $\frac{d(v)}{N}$. Thus, the
expected number of new connections generated by $v$ at this step is $\frac{d(v)}{
N}\frac{D}{d(v)}=\frac{D}{N}$.

Consider a time interval $[t- c\log N,t]$. W.h.p. the network has no more than
$N(1+o(1))$ nodes at each step of this interval. Thus, the expected number of
connections to the cache due to deletions during the whole interval is $D(1+o(1))c\log
N$. Each node that leaves the network contributes up to $C$ connections to this count.
We partition the count to $C$ sums,  each is a sum of independent $0-1$ events, and
apply the Chernoff bound to each of the sums to prove that the w.h.p the total number
of connection to cache nodes from old nodes is bounded by $D(1+o(1))c \log N$. The new
arrivals during this interval have Poisson distribution with expectation $c\log N$, and
each contribute $D$ connections to cache node. Thus, w.h.p we have $c\log N(1\pm o(1))$
new nodes and up to $D(1+o(1))c\log N$ connections from these nodes. A cache node
enters the cache with $D$ connections and leaves with $C$ connections, thus, w.h.p.
this interval generates up to $c\log N\frac{2D(1+o(1))}{C-D}\leq \frac{2}{3} c\log N$ new c-nodes,
and at least $\frac{1}{3}c \log N$ d-nodes.
Since w.h.p. at least $c\log N(1-o(1))$ d-node arrived in the interval, we have a total
of $\frac{1}{4}c\log N$ in the network at time $t$.
\end{proof}
\section{Figures and Pseudocode}

\subsection{Figures in Latex}

You can write figures in latex as in Figure \ref{fig:big-o}.
One way of creating figure is to draw using some software (e.g., MS Paint, Photoshop) and save it as a vector graphic (e.g. svg, pdf).
Then insert the file into the document using the figure environment. Here, Big-O.pdf and Big-O1.pdf are the figure files.
 
\begin{figure*}
    %\centering
    \includegraphics*[width=2in,height=2in]{figures/Big-O}
    \includegraphics*[width=2in, height=2in]{figures/Big-O1}
    \caption{Illustration of the Big-O notation. $g(n) = O(f(n))$.}
    \label{fig:big-o}
\end{figure*}

Alternatively, there are packages to draw figures directly. \href{https://tikz.dev/}{Tikz} is a popular one, though it has a high learning curve. For an example, see  \href{https://texample.net/tikz/examples/cycle/}{here}.



\subsection{Algorithms in Latex}

Pseudocode can be written in a few different ways. Using the \href{https://ctan.org/pkg/algpseudocodex}{algpseudocodex package}, Algorithm \nameref{alg:select} is given in \cref{alg:select}.

\begin{algorithm}
    \caption{Select}
    \label{alg:select}
    \begin{algorithmic}[1]
        \Function{\nameref*{alg:select}}{$S$, $k$}
        \If {$\card{S} \equalto 1$}
        \State \Return $S[1]$
        \EndIf
        \State Partition $S$ into $\ceil{\sfrac{n}{5}}$ groups of 5 elements each and a leftover group of up to 4 elements. \label{line:selectpartition}
        \State Find the median of each group \label{line:selectfindmedians}
        \State $R \gets$ the set of these $\ceil{\frac{n}{5}}$ medians \label{line:selectsetofmedians}
        \State $m \gets \card{R}$
        \State $p\gets \Call{Select}{R,\ceil{\sfrac{m}{2}}}$\label{line:selectmedianofmedians}
        \State $S_1 \gets \set{x\in S \suchthat x < p}$\label{line:selectleftpartition}
        \State $S_2 \gets \set{x\in S \suchthat x > p}$\label{line:selectrightpartition}
        \If {$\card{S_1} = k-1$}
        \State \Return $p$
        \ElsIf {$\card{S_1} > k - 1$}
        \State \Return $\Call{Select}{S_1 ,k}$
        \Else
        \State \Return $\Call{Select}{S_2,k-\card{S_1}-1}$
        \EndIf
        \EndFunction
    \end{algorithmic}
\end{algorithm}


\section{Citations}
Bibliographic information can be cited using \verb|\cite|. For example, \verb|\cite{algopandu}| cites the algorithm textbook \cite{algopandu}.
Typically, we keep bibliographic information in a \texttt{.bib} file (in this case, \texttt{references.bib}).

Internal references (with hyperlinks) can be created using \verb|\ref|, \verb|\nameref|, and \verb|\cref| (though \verb|\cref| comes from the
\href{https://ctan.org/pkg/cleveref}{Cleveref} package). The \verb|\ref| command will cite the corresponding numerical label (e.g.,
\verb|\ref{alg:select}| outputs ``\ref{alg:select}''). The \verb|\nameref| command will cite the name (if given), and \verb|\cref| will
output the numerical label with the type, e.g., \verb|\cref{alg:select}| outputs ``\cref{alg:select}''.

\bibliographystyle{abbrv}
\bibliography{references}
\end{document}

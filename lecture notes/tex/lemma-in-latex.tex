\section{A Lemma in Latex}

Lemmas are statements that help in proving a theorem (usually the main result).
Lemmas also should be stated correctly and proved. This Lemma is from one of my papers.

\begin{lemma}
\label{lem:capacity} Let $C> 4D$, then at any given step $t \geq c\log N$, for a fixed
constant $c >0$, w.h.p. there is at least $\frac{1}{4}c\log N$ d-nodes in the network.
Further, the above set of d-nodes arrived during the interval $[t-c\log N, t]$.
\end{lemma}

\begin{proof}
Let $v$ be a node of degree $d(v)$ at time $\tau$, then the expected number of
neighbors of $v$ that leave the network at time $\tau$ is $\frac{d(v)}{N}$. Thus, the
expected number of new connections generated by $v$ at this step is $\frac{d(v)}{
N}\frac{D}{d(v)}=\frac{D}{N}$.

Consider a time interval $[t- c\log N,t]$. W.h.p. the network has no more than
$N(1+o(1))$ nodes at each step of this interval. Thus, the expected number of
connections to the cache due to deletions during the whole interval is $D(1+o(1))c\log
N$. Each node that leaves the network contributes up to $C$ connections to this count.
We partition the count to $C$ sums,  each is a sum of independent $0-1$ events, and
apply the Chernoff bound to each of the sums to prove that the w.h.p the total number
of connection to cache nodes from old nodes is bounded by $D(1+o(1))c \log N$. The new
arrivals during this interval have Poisson distribution with expectation $c\log N$, and
each contribute $D$ connections to cache node. Thus, w.h.p we have $c\log N(1\pm o(1))$
new nodes and up to $D(1+o(1))c\log N$ connections from these nodes. A cache node
enters the cache with $D$ connections and leaves with $C$ connections, thus, w.h.p.
this interval generates up to $c\log N\frac{2D(1+o(1))}{C-D}\leq \frac{2}{3} c\log N$ new c-nodes,
and at least $\frac{1}{3}c \log N$ d-nodes.
Since w.h.p. at least $c\log N(1-o(1))$ d-node arrived in the interval, we have a total
of $\frac{1}{4}c\log N$ in the network at time $t$.
\end{proof}
\section{Theorems}



Theorems that we discuss in class should be stated correctly and proved. Here is an example.

\bigskip

Markov's inequality uses only the value of the expectation (also called ``first moment'').
Hence it can be quite weak.

\begin{theorem}[Markov Inequality]
    For any non-negative random variable and for any $a > 0$
    \[\prob{X\geq a}\leq \frac{\expectation{X}}{a}\]
\end{theorem}


Except for the easiest theorems and proofs, it is important to give an intuition for them.
Here is an example.

Intuitively, Markov's inequality says that for any (positive) random variable $X$, it is less likely  that
$X$ takes a value  much larger than its expectation. 



\begin{proof}
    For any $a > 0$, let
    $I$ be an indicator r.v. for the event  $X \geq a$.
    Then $I \leq X/a$. Taking expectations on both sides, we get   $\expectation{I} = \prob{X \geq a} \leq \frac{\expectation{X}}{a}$.
\end{proof}



Chebyshev's inequality gives a stronger bound, which  assumes that the \emph{variance} (also called the ``second moment'') or \emph{standard deviation} of the random variable is known.
These are defined below.


It is good to define essential concepts formally by defining them. Here is an example.

\begin{definition}
    The \textbf{variance} of a random variable $X$ is
    \[\variance{X} = \expectation{\left(X-\expectation{X}\right)^2}\]
    The \textbf{standard deviation} of a random variable $X$ is
    \[\sigma(X)=\sqrt{\variance{X}}\]
\end{definition}

Now we state and prove Chebyshev's inequality.

\begin{theorem}{Chebyshev's Inequality}
    For \textbf{any} random variable
    \[\prob{\abs{X-\expectation{X}} \geq a} \leq \frac{\variance{X}}{a^2}\]
\end{theorem}

The proof uses Markov's inequality. Note that Chebyshev's inequality applies to any random variable,
not necessarily positive. Thus to apply Markov (which applies only to positive r.v.) we 
apply it to $\abs{X-\expectation{X}}$. The usual trick to get rid of absolute value, which is hard to deal with, is by squaring.


\begin{proof}
    \[\prob{\abs{X-\expectation{X}} \geq a} = \prob{(X-\expectation{X})^2 \geq a^2}\]
    By Markov inequality,
    \[ \prob{(X-\expectation{X})^2 \geq a^2} \leq \frac{\expectation{\left(X-\expectation{X}\right)^2}}{a^2}=\frac{\variance{X}}{a^2}\]
\end{proof}

